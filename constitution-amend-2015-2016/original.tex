Ratified: April 2nd, 1990

Amendments: May 28th, 1992

\section*{Preamble}\label{preamble}

The Computer Science Undergraduate Society (CSUS) is only a legal body
within the compound of McGill University. This organization is strictly
a student organization and student interest group. The CSUS is a
non-profit organization and receives funding based on a formula as
described in the Science Undergraduate Society (SUS) and the Student
Society of McGill University (SSMU).

\part{The Society}\label{the-society}

\section{Name}\label{name}

The official name of the society, in EngIish, shall be the "Computer
Science Undergraduate Society of McGill University", and in French,
"L'Association des Etudiants et Etudiantes en Informatique de
L'Universite McGill", herein referred to as CSUS and AEIM respectively.

\section{Membership}\label{membership}

2.1 Membership of the CSUS shall be all students registered in an
Undergraduate programme in the School of Computer Science at McGill
University, subject to payment of fees prescribed in Article 4. For the
purpose of membership, any student registered in a Computer Science
Major or Computer Science Honours programme shall be considered a
member. Students registered in a Computer Science Minor or Computer
Science Joint Honours may be granted membership by the Executive Council
subject to Article 4 (payment).

\section{Purpose}\label{purpose}

The purpose of the CSUS is to represent and promote the views of its
members and to implement academic, educational, cultural, social and
other programmes of interest to its members.

\section{Society Fees}\label{society-fees}

The fees for the CSUS will be determined by the Science Undergraduate
Society of McGill University (Hereafter referred to as SUS).

\section{Rights, Privileges and Obligations of
Members}\label{rights-privileges-and-obligations-of-members}

\begin{enumerate}
\def\labelenumi{\arabic{enumi}.}
\tightlist
\item
  The rights of the members shall include the following:
\end{enumerate}

\begin{enumerate}
\def\labelenumi{(\alph{enumi})}
\tightlist
\item
  the right to vote in the CSUS general elections and referenda;
\item
  the right to attend General Assemblies and Executive Council meetings
  of the CSUS subject to the rules of procedure described in article 22;
\item
  the right to initiate referenda or General Assemblies according to
  articles 20 and 24;
\item
  the right to move or second a motion at any General Assembly;
\item
  the right to speak for or against any motion at any General Assembly;
\item
  the right to nominate candidates for Society elections according to
  the Electoral bylaws.
\end{enumerate}

\begin{enumerate}
\def\labelenumi{\arabic{enumi}.}
\setcounter{enumi}{1}
\tightlist
\item
  The privileges of the members shall include:
\end{enumerate}

\begin{enumerate}
\def\labelenumi{(\alph{enumi})}
\tightlist
\item
  holding office within the CSUS subject to the qualifications as
  specified in article 26;
\item
  making use of the CSUS facilities and services.
\end{enumerate}

\begin{enumerate}
\def\labelenumi{\arabic{enumi}.}
\setcounter{enumi}{2}
\item
  The obligations of all members of the CSUS will be to conform to the
  CSUS Constitution, regulations and bylaws.
\item
  Every regular member has the right to request from the Vice-President
  Administration that a question be asked at any General Assembly or
  referendum of the CSUS pursuant to the requirements of articles 20 and
  24 by submitting a petition bearing the signatures of twenty percent
  (20\%) of the members of the CSUS.
\item
  No member is empowered to make purchase in the name of the CSUS or to
  financially obligate the CSUS in any way, until such permission has
  been granted to so purchase or obligate the CSUS by the Executive
  Council of the CSUS.
\item
  No member is empowered to act as an agent of the CSUS unless
  permission to so act has been granted be the Executive Council of the
  CSUS.
\end{enumerate}

\section{Finances of the CSUS}\label{finances-of-the-csus}

\begin{enumerate}
\def\labelenumi{\arabic{enumi}.}
\item
  The fiscal year of the Society shall from April 15th to April 14th of
  the following year.
\item
  The budget of the CSUS for the current fiscal year will be presented
  to the SUS no later than October 1st.
\item
  The accounts of the CSUS shall be maintained according to standard
  accounting practice and shall be made available to the University
  auditors or any student on demand.
\end{enumerate}

\section{Languages of the
Society}\label{languages-of-the-society}

\begin{enumerate}
\def\labelenumi{\arabic{enumi}.}
\item
  English and French are the official languages of the Society.
\item
  At all meetings of the Society and its committees, members may use
  either official languages.
\item
  Resolutions of the Society and its committees may be adopted in either
  or both official languages.
\item
  Documents may be obtained from the CSUS in any particular language
  upon demand.
\end{enumerate}

\part{Organization of the
Society}\label{organization-of-the-society}

\section{The Executive Council}\label{the-executive-council}

The Executive Council shall generate, formulate and implement the
policies of the Society.

\section{Members of the Executive
Council}\label{members-of-the-executive-council}

\begin{enumerate}
\def\labelenumi{\arabic{enumi}.}
\tightlist
\item
  The Executive Council shall consist of:
\end{enumerate}

\begin{enumerate}
\def\labelenumi{(\alph{enumi})}
\tightlist
\item
  the President;
\item
  the Vice-President Internal;
\item
  the Vice-President External;
\item
  the Vice-President Finance;
\item
  the Vice-President Academic;
\item
  the Vice-President Administration;
\item
  the U1 Representative.
\end{enumerate}

\begin{enumerate}
\def\labelenumi{\arabic{enumi}.}
\setcounter{enumi}{1}
\item
  Members of the Executive Council must be members of the CSUS for the
  entire duration of their mandate. No exception shall be granted.
\item
  No member of the Executive Council shall receive financial
  remuneration for the fulfillment of their mandate.
\item
  No student having been convicted for a criminal offense, internal or
  external of the University shall be allowed to take office. For the
  purpose of internal, a conviction by the University's Judiciary board;
  For the purposes of external, a conviction by any court of the land,
  be it municipal, provincial, Superior, Federal or the Supreme Court of
  Canada and all Court of Appeals.
\end{enumerate}

\section{Powers and Duties}\label{powers-and-duties}

\begin{enumerate}
\def\labelenumi{\arabic{enumi}.}
\tightlist
\item
  The Executive Council shall:
\end{enumerate}

\begin{enumerate}
\def\labelenumi{(\alph{enumi})}
\tightlist
\item
  define all general policies of the CSUS;
\item
  coordinate and administer the policies, activities and other day to
  day affairs of the CSUS;
\item
  act as the governing body of the CSUS, empowered to make all decisions
  and take responsible and required actions on behalf of the CSUS;
\item
  be held accountable to General Assembly and ensure the execution of
  General Assembly decisions;
\item
  approve and/or reject budgets for all the CSUS Committees; This task
  may be delegated to the Vice-President Finance with the approval of
  2/3 majority of members present and voting.
\item
  create or dissolve CSUS Committees;
\item
  uphold the Constitution, regulations, policies and bylaws of the CSUS;
\item
  determine the membership fee of each member of the Society, subject to
  ratification by referendum according to article 24;
\item
  appoint the Chief Returning Officer.
\end{enumerate}

\begin{enumerate}
\def\labelenumi{\arabic{enumi}.}
\setcounter{enumi}{1}
\tightlist
\item
  Signing powers of the Society shall be exercised by the following, any
  two of whom shall be required for any given transaction:
\end{enumerate}

\begin{enumerate}
\def\labelenumi{(\alph{enumi})}
\tightlist
\item
  the President;
\item
  the Vice-President Finance;
\item
  the Vice-President Administration.
\end{enumerate}

\begin{enumerate}
\def\labelenumi{\arabic{enumi}.}
\setcounter{enumi}{2}
\item
  Each member of the Executive Council shall prepare and submit to the
  Executive Council a year-end report no later than April 14th.
  (Although it is recommended that a semester-end report should also be
  given to inform the constituence of the going ons of the council)
\item
  A member of the Executive Council shall cease to remain in office upon
  acceptance of their letter of resignation, or upon their impeachment.
\end{enumerate}

\section{President}\label{president}

The President shall:

\begin{enumerate}
\def\labelenumi{\arabic{enumi}.}
\item
  coordinate and supervise the affairs of the Society;
\item
  call and chair over the Executive Council meetings;
\item
  chair General Assemblies;
\item
  serve as an ex-officio member of all Society committees;
\item
  be the official representative of the Society.
\end{enumerate}

\section{Vice-President
External}\label{vice-president-external}

The Vice-President External shall:

\begin{enumerate}
\def\labelenumi{\arabic{enumi}.}
\item
  in the absence of the President, be empowered to perform any function
  of the President;
\item
  be responsible for maintaining links with student organizations at the
  university, provincial, federal and international levels and with
  computer science student societies of other universities;
\end{enumerate}

\begin{enumerate}
\def\labelenumi{\arabic{enumi}.}
\setcounter{enumi}{3}
\tightlist
\item
  be responsible for maintaining relations with industry, government and
  other groups outside University.
\end{enumerate}

\section{Vice-President
Internal}\label{vice-president-internal}

The Vice-President Internal shall:

\begin{enumerate}
\def\labelenumi{\arabic{enumi}.}
\tightlist
\item
  be responsible for the organization of social, cultural and other
  activities for the members of the CSUS;
\end{enumerate}

\begin{enumerate}
\def\labelenumi{(\alph{enumi})}
\tightlist
\item
  The Vice-President Internal shall be chair of the Social Activities
  Committee (SAC).
\end{enumerate}

\begin{enumerate}
\def\labelenumi{\arabic{enumi}.}
\setcounter{enumi}{1}
\item
  maintain and promote relations with other Faculties, Student
  Associations and administrative bodies of the University (Internal
  Affairs).
\item
  be responsible for the society's facilities and equipment.
\end{enumerate}

\section{Vice-President Finance}\label{vice-president-finance}

The Vice-President Finance shall:

\begin{enumerate}
\def\labelenumi{\arabic{enumi}.}
\item
  in cooperation with the Executive Council, prepare the annual budget
  of the CSUS, which shall include the actual expenditures from the
  previous year, before October 15;
\item
  in cooperation with the Executive Council, manage the funds of the
  CSUS;
\item
  keep proper financial accounts and records;
\item
  prepare a year-end financial report by April 14th for review by an
  independent person at minimal cost to the society.
\item
  present a complete semester-end financial report that shall be made
  public and shall be readily available to the constituents.
\end{enumerate}

\section{Vice-President
Academic}\label{vice-president-academic}

The Vice-President Academic shall:

\begin{enumerate}
\def\labelenumi{\arabic{enumi}.}
\item
  be responsible for all educational and curricular concerns of the
  CSUS, whether they be internal or external to the University.
\item
  (may) represent a student, upon the demand of the student in writing,
  in any judicial or academic or social proceedings taken against the
  student by the University, or a delegate appointed by the University.
\item
  be chair of the University Academic Committee (UAC).
\end{enumerate}

\section{Vice-President
Administration}\label{vice-president-administration}

The Vice-President Administration shall:

\begin{enumerate}
\def\labelenumi{\arabic{enumi}.}
\item
  be responsible for preparing and issuing agendas and minutes of the
  CSUS Executive Council meetings and General Assemblies at least three
  (3) school days prior to any CSUS Executive Council meeting or General
  Assembly;
\item
  promote and coordinate communication within the Society;
\item
  maintain the files of the Society;
\item
  ensure members of the CSUS Executive Council attend meetings;
\item
  be responsible to have official minutes of the Executives and the
  General Assembly readily available on demand.
\item
  serve as an ex-officio member of all Society committees;
\end{enumerate}

\section{U1 Representative}\label{u1-representative}

The U1 Representative shall:

\begin{enumerate}
\def\labelenumi{\arabic{enumi}.}
\item
  act as liaison between the U1 students and the Executive Council;
\item
  represent the views of the Ul students at meetings of the Executive
  Council;
\item
  hold a meeting of the Ul students when necessary.
\item
  serve on the University Academic Committee (UAC) as Vice-Chairperson.
\item
  serve as the Vice-Chair of the Executive Council.
\end{enumerate}

\section{Meetings of the Executive
Council}\label{meetings-of-the-executive-council}

\begin{enumerate}
\def\labelenumi{\arabic{enumi}.}
\item
  The CSUS Executive Council shall hold meetings at least once every two
  weeks while classes are in session.
\item
  Quorum for a Regular meeting of the CSUS Executive Council shall be
  four (4) members.
\item
  Each member of the Executive Council shall have a single vote, except
  for the President who shall only vote to break a tie.
\item
  If the Chairperson wishes to add his comment(s) to the discussion,
  he/she must first pass the chair to the next executive (in order of
  succession). {[}Definition: Comments - Any statement(s) that may have
  any bearing to the discussion; excluded are Chairperson's remarks
  about (see Robert's Rule of Order), and decisions of the Chair.
\item
  In case of an emergency, the President may call a special meeting at
  any time, provided that all executives are present, or upon the
  signing of a waiver of notice by all executives.
\end{enumerate}

\section{Powers of Assembly}\label{powers-of-assembly}

The General Assembly may make any decision, including the ratification
or rejection of any Executive Council decision. The General Assembly may
also impeach a (the) member(s) of the Executive Council as prescribed
under Article 25 and the Impeachment Annex.

\section{Meetings of the General
Assembly}\label{meetings-of-the-general-assembly}

\begin{enumerate}
\def\labelenumi{\arabic{enumi}.}
\item
  The General Assembly may be called by a resolution of the Executive
  Councilor by a petition signed by at least twenty percent (20\%) of
  the members of the Society. (It is advised that at least one General
  Assembly be held once per semester as to allow students to become
  involved with the affairs of the Society).
\item
  A notice announcing the meeting of the General Assembly shall be
  posted at least five (5) teaching days prior to the convening of the
  General Assembly. The notice must be posted in areas where the maximum
  number of CS students will be able to see.
\item
  Motions to be presented at the General Assembly must be submitted to
  the Vice-President Administration at least three (3) school days prior
  to the General Assembly. However motions may be made at a General
  Assembly by any student after having secured the floor.
\item
  The order of business and motions to be presented at the General
  Assembly shall be posted two (2) school days prior to the General
  Assembly. The notices must be placed in areas where the maximum number
  of CS students will be able to see.
\item
  The Quorum for the General Assembly shall be twenty percent (20\%) of
  the members of the Society.
\item
  Procedure at the General Assembly shall be subject to the rules of
  order stated in article 22.
\item
  A General Assembly of the CSUS shall be held at least twice every
  school year. (Recommended: once every semester).
\end{enumerate}

\section{Committees of the CSUS}\label{committees-of-the-csus}

\begin{enumerate}
\def\labelenumi{\arabic{enumi}.}
\item
  Any member of the CSUS is eligible to hold chair on a CSUS Committee,
  unless otherwise specified.
\item
  Each Committee shall be responsible for submitting a year-end report
  to the Executive Council no later than April 1st. (Although it is
  recommended that a weekly and/or a semester-end report be given to the
  Executive Council).
\item
  A Committee shall be defined by its own separate bylaws, subject to
  approval by the Executive Council.
\item
  The CSUS may, as it sees fit, establish Committees to assist the
  Executive Council. No member shall be renumerated for their services
  render, unless authorized by the Executive Council.
\item
  The creation or dissolution of a Committee or the modification of its
  bylaws shall require a resolution with two-thirds (2/3) majority
  present and voting in the Executive Council.
\end{enumerate}

\section{Procedure}\label{procedure}

The rules of procedure for all meetings of the Executive Council and the
General Assembly shall be the most recent edition of Roberts' Rules of
Order. (Note: An annex must be inserted to the Constitution if a
permanent change is made to Roberts' Rules of Order).

\section{Electoral Officer}\label{electoral-officer}

\begin{enumerate}
\def\labelenumi{\arabic{enumi}.}
\item
  The Chief Returning Officer (CRO) shall be responsible for all aspects
  of the administration of Society elections and referenda according to
  the Electoral bylaws of the CSUS.
\item
  The CRO may not be a candidate in any Society election; if he wishes
  to be a candidate for any elected Society position, the CRO shall
  resign from the position of CRO. The Executive Council shall then
  elect a new CRO subject to Article 28.
\item
  The CRO may be removed from office for dereliction of duties as
  specified in Article 25.
\end{enumerate}

\section{Referenda}\label{referenda}

\begin{enumerate}
\def\labelenumi{\arabic{enumi}.}
\item
  A Referendum may be initiated by a resolution of the Executive Council
  or a petition signed by at least twenty percent (20\%) of the members
  of the Society. The requirements for impeachment are described in
  article 25.
\item
  Notice of the referendum question, voting location and voting hours
  must be posted no less than six (6) school days before the vote is to
  take place. All notices must be posted in highly public area,
  especially areas frequented by CS students.
\item
  The referendum shall be held according to the Electoral bylaws of the
  CSUS.
\item
  The Referendum shall be considered valid only if a minimum of thirty
  percent (30\%) of the CSUS members vote.
\item
  A simple majority is required to pass a referendum. The requirements
  for constitutional amendments are stated in article 31.
\item
  The result of a Referendum shall be binding on the Society and take
  precedence over decisions of the Executive Council and the General
  Assembly.
\end{enumerate}

\section{Impeachment}\label{impeachment}

\begin{enumerate}
\def\labelenumi{\arabic{enumi}.}
\tightlist
\item
  A member of the Executive Council may be removed by way of referendum
  initiated by either:
\end{enumerate}

\begin{enumerate}
\def\labelenumi{(\alph{enumi})}
\tightlist
\item
  two-thirds (2/3) majority vote of the Executive Council;
\item
  a petition signed by twenty-five percent (25\%) of the members of his
  or her constituency.
\end{enumerate}

\begin{enumerate}
\def\labelenumi{\arabic{enumi}.}
\setcounter{enumi}{1}
\item
  The CRO may be relieved of his or her duties by a two-thirds (2/3)
  vote of the Executive Council.
\item
  Impeachment procedures to be followed are specified in Annex 1 : IMPEACHMENT Proceedings.
\end{enumerate}

\part{Elections}\label{elections}

\section{Eligible Voters and
Candidates}\label{eligible-voters-and-candidates}

\begin{enumerate}
\def\labelenumi{\arabic{enumi}.}
\tightlist
\item
  The following shall be elected by and from the regular members of the
  CSUS: (a)the President; (b)the Vice-President Internal;
\end{enumerate}

\begin{enumerate}
\def\labelenumi{(\alph{enumi})}
\setcounter{enumi}{2}
\tightlist
\item
  the Vice-President External; (d)the Vice-President Finance; (e)the
  Vice-President Academic; (f)the Vice-President Administration.
\end{enumerate}

\begin{enumerate}
\def\labelenumi{\arabic{enumi}.}
\setcounter{enumi}{1}
\item
  The U1 Representative shall be elected by and from the regular members
  who are in their first year of studies in a computer science
  programme.
\item
  The Chief Returning Officer may not be candidate in any election of
  the CSUS.
\end{enumerate}

\section{Procedures}\label{procedures}

\begin{enumerate}
\def\labelenumi{\arabic{enumi}.}
\item
  Elections for all members of the Executive Council except the Ul
  Representative shall be held between March 1st and March 31st.
\item
  Elections for the Ul Representative shall be held during the month of
  September.
\item
  The elections shall be held according to the Bylaws of the CSUS.
\end{enumerate}

\section{Filling of Vacancies}\label{filling-of-vacancies}

In the event of an Executive vacancy, the Executive Council shall
instruct the CRO to hold a by-election accord ing to the Electoral
bylaws of the CSUS.

\section{Terms of Office}\label{terms-of-office}

\begin{enumerate}
\def\labelenumi{\arabic{enumi}.}
\tightlist
\item
  The terms of office for all members of the Executive Council except
  the Ul Representative shall begin on the 15th of April and last one
  year.
\item
  The terms of office for the Ul Representative shall be from the day he
  or she is elected to April 14th of the following calendar year.
\end{enumerate}

\section{Order of Succession}\label{order-of-succession}

\begin{enumerate}
\def\labelenumi{\arabic{enumi}.}
\tightlist
\item
  For the purpose of continuity, in the case of prolonged absence,
  illness, resignation, impeachment or death of an executive, the
  following options are available:
\end{enumerate}

\begin{enumerate}
\def\labelenumi{(\alph{enumi})}
\tightlist
\item
  Another executive may assume the duties;
\item
  A by-election subject to the Election By-laws.
\item
  Hiring of an Executive Director to complete the mandate of the
  executive in question. No member of the Executive may be hired or
  renumerated for their services rendered.
\end{enumerate}

\begin{enumerate}
\def\labelenumi{\arabic{enumi}.}
\setcounter{enumi}{1}
\tightlist
\item
  In the event of prolonged absence or illness, resignation, impeachment
  or death of the President of the CSUS, the order of succession shall
  be as follows:
\end{enumerate}

\begin{enumerate}
\def\labelenumi{(\alph{enumi})}
\tightlist
\item
  Vice-President External;
\item
  Vice-President Finance;
\item
  Vice-President Internal;
\item
  Vice-President Administration;
\item
  Vice-President Academic;
\item
  U1 Representative.
\item
  C.R.O.
\end{enumerate}

By definition, a person is incapacitated when he/she is no longer able
to perform the duties as specified in the Society's Constitution, and is
unable to continue their mandate.

The C.R.O. is listed, in the event none of the members of the Executive
Council is able to assume the office of President, shall be the acting
President until such time as an election or by-election is held to fill
all vacancies as prescribed by the Society's Constitution and Electorate
Bylaws.

\part{The Constitution}\label{the-constitution}

\section{Superseding Clause}\label{superseding-clause}

This Constitution supersedes and repeals all previous Constitutions of
the Society.

\section{Constitutional
Amendments}\label{constitutional-amendments}

The Constitution of the society may only be amended by a referendum in
accordance to Article 24 with a majority of two-thirds (2/3) of the
members voting in favour.

\section{Coming into force}\label{coming-into-force}

The Constitution shall come into force April 4th, 1990.

Amendment: The Constitution shall come into force upon the Declaration
of the Executives subject to Article 32.

\part{Bylaws}\label{bylaws}

\section{A Electoral Bylaws}\label{a-electoral-bylaws}

\paragraph{A.1 Nomination Rules}\label{a.1-nomination-rules}

\begin{enumerate}
\def\labelenumi{\arabic{enumi}.}
\item
  The CRO shall post, at an appropriate time, giving notice to all
  students, a list of positions open to nomination and election along
  with a time for the opening and closing of the nominations.
\item
  The closing date for the nominations will be no later than five (5)
  school days before the elections.
\item
  All nominations must contain the words \textgreater{} ``We, the
  undersigned, nominate \_ for the position of \_ for the 20(n)-20(n+1)
  academic year.''
\item
  With respect to the elections of the U1 representative, all
  nominations must be signed by five (5) students eligible to vote
  according to article 26.
\item
  With respect to the general elections, all nominations must be signed
  by ten (10) students eligible to vote according to article 26.
\item
  All nominations must be presented to the CRO before the closing dead
  line established by the CRO.
\item
  If at the close of nominations, any position is such that it would
  result in a vacancy or acclamation, the CRO shall re-open nominations
  for one (1) additional teaching day.
\item
  The CRO shall validate the nominations and publicize them within
  twelve (12) hours of the closing of the nominations.
\end{enumerate}

\paragraph{A.2 Campaigning Rules}\label{a.2-campaigning-rules}

\begin{enumerate}
\def\labelenumi{\arabic{enumi}.}
\item
  The CRO shall post the date, time and location of the elections at
  least five (5) school days before the elections.
\item
  The campaign period for all candidates will commence one week before
  the election and will end at 23:00 hrs. of the day before the
  election.
\item
  The posting of notices and signs must conform to the general rules of
  McGill University concerning the placing of such materials on the
  University premises.
\item
  The CRO may designate rules from time to time with respect to the
  placing of such materials in the University.
\item
  The CRO may set a maximum spending limit of which the CSUS will
  reimburse each candidate a predetermined amount. (Ensure that all
  candidates get to spend the same amount).
\item
  All receipts (except for the use of an automobile and public
  transport) must be turned over to the CRO before the end of the voting
  period. The CRO has the right to refuse a (all) receipt(s) that are
  not deemed to be proper, official or otherwise.
\end{enumerate}

\paragraph{A.3 Balloting Rules}\label{a.3-balloting-rules}

\begin{enumerate}
\def\labelenumi{\arabic{enumi}.}
\tightlist
\item
  The CRO shall be responsible for appointing polling clerks.
\end{enumerate}

\begin{enumerate}
\def\labelenumi{(\alph{enumi})}
\tightlist
\item
  Polling clerks shall be paid the minimum wage/hour, or a predetermined
  amount not to be less that the (hours * minimum wage) formula.
\end{enumerate}

\begin{enumerate}
\def\labelenumi{\arabic{enumi}.}
\setcounter{enumi}{1}
\item
  Balloting shall be done in accordance to article 26.
\item
  Before polls open, the CRO shall cast his vote and seal it under guard
  of the President and Vice-President Administration of the Society. The
  vote of the CRO shall only be opened in the case of a tie in which
  case it will only be used to break a particular tie(s).
\end{enumerate}

\paragraph{A.4 Count, Recount and
Protests}\label{a.4-count-recount-and-protests}

\begin{enumerate}
\def\labelenumi{\arabic{enumi}.}
\item
  Ballots shall be counted as soon as practicable after the closing of
  the polls under the supervision of the CRO and those whom the CRO
  designate to assist him.
\item
  No ballot shall be counted in the presence of less than two persons.
\item
  Ballot shall be rejected if they are deemed spoiled by the CRO.
\item
  When the counting of the ballots has been completed, the CRO will
  announce the results to the candidates. The results shall then be
  posted publicly.
\item
  All complaints, protest or petitions for a recount must be made to the
  CRO no later than three (3) school days following the closing of the
  polls.
\end{enumerate}

\paragraph{A.5 Invalidation}\label{a.5-invalidation}

The CRO shall invalidate the election if upon investigation it is
evident that there was a gross violation of these bylaws such as to:

\begin{enumerate}
\def\labelenumi{\arabic{enumi}.}
\item
  disenfranchise eligible voters;
\item
  permit ineligible persons to vote;
\item
  mislead voters in their choice;
\item
  interfere with voting on election days.
\item
  groups running as a party (be it official or unofficial). If a group
  of candidates decide to run as a collective, then the CRO shall give
  notice that such action is illegal, as well, as advertisement with the
  names of collective candidates.
\end{enumerate}

\section{B Bylaw Amendments}\label{b-bylaw-amendments}

Amendments to the by-laws may be made at any meeting of the CSUS
Executive Council and must be approved by two-thirds (2/3) of those
present and voting.

\begin{center}\rule{0.5\linewidth}{\linethickness}\end{center}

\part{ANNEX 1 : Impeachment}\label{annex-1-impeachment}

Any member of the Executive Council who is derelict of duty shall be
subject to impeachment proceedings upon ten (10) academic days written
notice by the Executive Council if the day, time and place of the
proceedings, to be delivered personally or by prepaid registered mail.
For the purpose of this section, a member of the Executive shall be
deemed to be derelict of duty if :

\begin{enumerate}
\def\labelenumi{\alph{enumi})}
\item
  (s)he is elected in contravention of the By-Laws and/or legally
  adopted proceedings of the Society;
\item
  once elected, (s)he knowingly contravenes the By-Lays and/or legally
  adopted proceedings of the Society;
\item
  (s)he fails to fulfill the duties imposed upon him/her by the By-Laws
  and/or legally adopted proceedings of the Society;
\item
  (s)he contravenes the Society's code of ethics in Annex B;
\item
  (s)he fails to attend two out of three consecutive scheduled meetings
  of the Executives without having given prior acceptable notice of
  absence, or displays a repeated and marked tendency to be absent from
  meetings of the Society.
\end{enumerate}

The subject of the impeachment procedures, as well as the complaint(s)
shall each and all be entitled present their respective cases before the
Society prior to the impeachment vote. The subject of the proceedings
may only be impeached by a two-thirds (2/3) majority of those eligible,
present and voting, excluding the Chairperson. In the event of a tie,
the Chair shall vote to break the tie.

\section{IMPEACHMENT PROCEDURES}\label{impeachment-procedures}

The impeachment proceedings shall be as follows:

\begin{enumerate}
\def\labelenumi{\alph{enumi})}
\item
  All proceedings shall be subject to Article 11 (a to i) of the
  Canadian Charter of Rights and Freedom (concerning the rights of the
  accused). The proceedings shall be in either of the official
  languages.
\item
  On the appointed day of the proceedings, the Chairperson of the
  proceedings will summarize the list of charges that have been filed
  against the individual (hereafter, the accused shall be refereed to as
  the ``individual'').
\item
  The Chairperson of the proceedings shall be appointed by the Executive
  Council prior to the proceedings. The VP Administration shall be the
  secretary of the proceedings. In the event the VP Administration is
  being impeached, the Chair shall designate a Secretary from amongst
  the members of the Society. The individual may not be the chair of
  his/her proceedings, nor is (s)he allowed to be the secretary of the
  proceedings. The individual shall not be allowed to vote at such
  proceedings.
\item
  The proceedings will be to determine if the individual is guilty or
  not guilty of the charge(s). Evidence shall be presented at the
  hearing. Anyone testifying as a witness against the individual shall
  be subjected to Article 13 of the Canadian Charter of Rights and
  Freedoms (protection from self incrimination).
\end{enumerate}

\begin{center}\rule{0.5\linewidth}{\linethickness}\end{center}

\part{Recommendation}\label{recommendation}

\begin{enumerate}
\def\labelenumi{\arabic{enumi}.}
\item
  Involvement of the general student population in worthy projects such
  that it is beneficial to the students. being a small faculty, the well
  being of the collective depends on the well being of the individual.
\item
  Set out and accomplish at least 5 major goals within an academic year.
  Students would like to see action and result, not bureaucracy nor
  administrative delays. With a pending event, project, the focus should
  be on that specific event, rather than split it up to those
  responsible for the activity.
\item
  Better forms of communication, be it electronic, visual, or open air
  presentation. Students would like to know what's going on, but
  sometimes don't have the time. Go in person to a classroom (with
  permission from prof of course). Publish a newsletter or put up
  posters advertising an event well in advance rather than last minute.
\end{enumerate}
